\documentclass{article}
\usepackage{graphicx} % Required for inserting images
\usepackage{setspace} % Double Space 
\usepackage{ragged2e} % justify text
\usepackage{enumitem} % enumeration within subsection
\usepackage[letterpaper, left=2cm, right=2cm, top=3cm, bottom=3cm]{geometry}

\title{Tesis}
\author{GUSTAVO ANDRES GONZALEZ PINEDA}
\date{February 2025}

\begin{document}

\begin{center}
\begin{doublespace}
    \Large{UNIVERSIDAD DEL VALLE DE GUATEMALA}\\
    Facultad de Ingeniería \\
    Ingeniería en Ciencias de la Computación y Tecnologías de la Información 

    % Image placement
    \vspace{15mm} 
    \includegraphics[width=0.2\textwidth]{images/Uvg_logo.jpg}

    \vspace{15mm} 
    {\Large Desarrollo de algoritmo de Pathfinding en Aplicación de recorridos virtuales con realidad aumentada para el Centro de Innovación y Tecnología de la Universidad del Valle de Guatemala.}

    \vspace{10mm} 
    {\Large Trabajo de graduación en modalidad de Megaproyecto presentado por Gustavo González
    para optar al grado academico de Licenciatura en Ingeniería en Ciencias de la Computación y Tecnologías de la Información.}

    \vspace{10mm} 
    {\Large Guatemala, \\ 2025}
    
\end{doublespace}
\end{center}

\newpage
\section{Objetivos}
\subsection{Objetivo General}
{\justify Desarrollar e implementar un sistema de localización y navegación basado en realidad aumentada para la Universidad 
del Valle de Guatemala, utilizando sensores Estimote Beacons con tecnología UWB y mejorando el algoritmo de pathfinding para 
optimizar la experiencia del usuario.}

\subsection{Objetivos Específicos}
\begin{enumerate}[label=\thesubsection.\arabic*]
    \item Realizar el mapeo de la universidad para determinar la ubicación óptima de los sensores Estimote Beacons con tecnología UWB.
    \item Configurar e integrar los sensores Estimote Beacons en la aplicación de realidad aumentada para mejorar la precisión de la 
    localización dentro del campus.
    \item Optimizar el algoritmo de pathfinding para generar rutas accesibles y mejorar la experiencia del usuario al navegar por la 
    universidad.
\end{enumerate}

\end{document}
