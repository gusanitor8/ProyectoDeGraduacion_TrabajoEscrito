\documentclass{article}
\usepackage{graphicx} % Required for inserting images
\usepackage{setspace} % Double Space 
\usepackage{ragged2e} % justify text
\usepackage{enumitem} % enumeration within subsection
\usepackage[letterpaper, left=2cm, right=2cm, top=3cm, bottom=3cm]{geometry}

\setlength{\parskip}{1em} % we increase paragraph spacing

\title{Tesis}
\author{GUSTAVO ANDRES GONZALEZ PINEDA}
\date{February 2025}

\begin{document}

\begin{center}
\begin{doublespace}
    \Large{UNIVERSIDAD DEL VALLE DE GUATEMALA}\\
    Facultad de Ingeniería \\
    Ingeniería en Ciencias de la Computación y Tecnologías de la Información 

    % Image placement
    \vspace{15mm} 
    \includegraphics[width=0.2\textwidth]{images/Uvg_logo.jpg}

    \vspace{15mm} 
    {\Large Desarrollo de algoritmo de Pathfinding en Aplicación de recorridos virtuales con realidad aumentada para el Centro de Innovación y Tecnología de la Universidad del Valle de Guatemala.}

    \vspace{10mm} 
    {\Large Trabajo de graduación en modalidad de Megaproyecto presentado por Gustavo González
    para optar al grado academico de Licenciatura en Ingeniería en Ciencias de la Computación y Tecnologías de la Información.}

    {\Large Guatemala, \\ 2025}
    
\end{doublespace}
\end{center}

\newpage
\section{Objetivos}
\subsection{Objetivo General}
{\justify Desarrollar e implementar un sistema de localización y navegación basado en realidad aumentada para la Universidad 
del Valle de Guatemala, utilizando sensores Estimote Beacons con tecnología UWB y mejorando el algoritmo de pathfinding para 
optimizar la experiencia del usuario.}

\subsection{Objetivos Específicos}
\begin{enumerate}[label=\thesubsection.\arabic*]
    \item Realizar el mapeo de la universidad para determinar la ubicación óptima de los sensores Estimote Beacons con tecnología UWB.
    \item Configurar e integrar los sensores Estimote Beacons en la aplicación de realidad aumentada para mejorar la precisión de la 
    localización dentro del campus.
    \item Optimizar el algoritmo de pathfinding para generar rutas accesibles y mejorar la experiencia del usuario al navegar por la 
    universidad.
\end{enumerate}

\section{Introducción}
{\justify
La orientación dentro de un campus universitario puede representar un desafío para estudiantes nuevos, visitantes y personas con
 discapacidad. Con el avance de la tecnología, la realidad aumentada (AR) se ha convertido en una herramienta innovadora para mejorar
  la experiencia de navegación en espacios físicos. En este contexto, el presente proyecto busca desarrollar una aplicación de realidad
   aumentada para recorridos dentro de la Universidad del Valle de Guatemala (UVG), proporcionando una guía interactiva e inclusiva 
   que facilite el desplazamiento desde un punto A hasta un punto B dentro del campus.

Este proyecto se basa en el trabajo realizado en años anteriores, donde se implementó una versión preliminar de la navegación con AR,
 pero con limitaciones significativas. Una de las principales dificultades fue la dependencia de códigos QR para la ubicación, lo que
  restringía la precisión y fluidez del recorrido. Además, la implementación del algoritmo de pathfinding presentaba problemas en la 
  generación de rutas óptimas, afectando la experiencia del usuario.

Para mejorar estos aspectos, la universidad ha invertido en sensores Estimote Beacons con tecnología Ultra-Wideband (UWB), los cuales
 permitirán una localización más precisa dentro del campus. La labor de este trabajo se centrará en el mapeo del campus para 
 determinar la ubicación óptima de estos sensores, su configuración e integración con la aplicación y la optimización del algoritmo
  de pathfinding para generar rutas más accesibles y eficientes.

El objetivo es mejorar la precisión de la navegación y evitar rutas poco prácticas o inaccesibles para ciertos usuarios. Esto 
contribuirá a una experiencia más fluida y efectiva al desplazarse dentro del campus, garantizando que la aplicación proporcione 
indicaciones precisas y usables en diferentes escenarios.A través de este trabajo, se espera ofrecer una solución innovadora y 
funcional que aproveche las capacidades de la realidad aumentada y la localización UWB para facilitar la movilidad dentro de la UVG,
 mejorando la orientación y accesibilidad dentro del campus universitario.}

\end{document}
